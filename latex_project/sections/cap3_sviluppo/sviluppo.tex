%****************************************************************%
% FILE: sviluppo.tex                                             %
% strumenti utilizzati, import dati, api                         %
%****************************************************************%
\documentclass[./main.tex]{subfiles}

\begin{document}
Questo capitolo si occupa di analizzare lo sviluppo di \textit{Ismar Data}. Inizialmente vengono presentati gli strumenti di produzione e comunicazione, con approfondimento sulle tecnologie software utilizzate nella realizzazione del back-end dell'applicazione. In seguito sarà descritta la fase di importazione dei dati dalle sorgenti disponibili. Si noterà come, nonostante il diverso tipo di formato dei dati, ci sarà il tentativo di creare un codice il più generale possibile; a volte sarà possibile, altre volte no. Infine si analizzeranno le API, sia quelle di autenticazione, importanti per la sicurezza, che quelle che si interfacciano con il front-end. Tutto il lavoro fatto viene supportato da alcune immagini provenienti da una demo web, usata come provvisoria soluzione in attesa dello sviluppo delle reali interfacce grafiche, che mostreranno i risultati ottenuti dall'elaborazione dei dati e dall'utilizzo delle API. 

\section{Strumentazione}
\subfile{/strumentazione.tex}

\section{Importazione dati}
\subfile{/import.tex}

\section{Costruzione API}\label{sec:api}
\subfile{/api.tex}
\end{document}