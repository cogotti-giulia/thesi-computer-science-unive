%****************************************************************%
% FILE: app.tex                                                  %
% funzionalità, architettura e sorgenti di dati app              %
%****************************************************************%
\documentclass[./main.tex]{subfiles}

\begin{document}
In questo capitolo verrà trattata \textit{Ismar Data}. Inizialmente saranno analizzate le funzionalità offerte all'utente finale. Un mockup grafico, ovvero una serie di immagini create appositamente per rappresentare le schermate dell'applicazione, aiuterà nel mostrare i risultati che si vorrebbero ottenere tramite l'implementazione delle diverse funzionalità. In seguito verrà introdotta l'architettura ideale dell'applicazione; ideale perché, nel corso di questo percorso, si noterà che alcuni intoppi renderanno difficile rispecchiarla nella realtà. Infine sarà dato ampio spazio alle sorgenti di dati, descrivendole e analizzando i dati che esse forniscono e le modalità con le quali ci si può accedere. Questa analisi finale porterà alla luce il problema della mancanza dei dati e della conseguente difficoltà a procedere nella produzione.

\section{Funzionalità}
\subfile{/funzionalita.tex}

\section{Architettura}
\subfile{/architettura.tex}

\section{Sorgenti di dati}
\subfile{/sorgenti_dati.tex}

\end{document}