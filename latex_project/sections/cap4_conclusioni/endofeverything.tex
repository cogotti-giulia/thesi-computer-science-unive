\documentclass[./main.tex]{subfiles}

\begin{document}

In questa tesi si è esaminato lo sviluppo di \textit{Ismar Data}, applicazione meteorologica creata da Elan42 e commissionata da ISMAR-CNR. Sono state analizzate le funzionalità richieste dal cliente e l'architettura ideale per realizzarle. Ampio spazio è stato dato all'analisi delle sorgenti di dati, pubbliche e private, con i relativi problemi di accesso ai rispettivi dati. Infine si è dato rilievo alla produzione vera e propria del prodotto, illustrando gli strumenti utilizzati e la fase di importazione dei dati. Ora sorge spontaneo porsi la seguente domanda: alla fine di questo percorso, \textit{sono stati raggiunti gli obiettivi?} Una risposta negativa è esattamente quella che ci si aspetta dopo aver letto il testo, e infatti è andata così. Sicuramente l'applicazione tratta un tema, quello meteorologico, poco familiare all'azienda produttrice. Questo fattore ha alzato inizialmente il livello di difficoltà, ma, dopo qualche settimana di studio il grado è diminuito drasticamente. Quindi, \textit{cosa non ha funzionato?} Ricordando il problema della mancanza dei dati, ampiamente discusso in questa tesi, dovuto al fatto che non sono stati forniti gli accessi alle sorgenti private, si ha immediatamente una risposta chiara: non ha funzionato la comunicazione e collaborazione tra i vari enti proprietari. Infine, \textit{come si sarebbe potuto risolvere il problema?} Dato che i tentativi di accordarsi non hanno funzionato, l'unica soluzione sarebbe stata prevenire il problema. ISMAR-CNR avrebbe dovuto assicurarsi, prima di incaricare Elan42 della produzione, che ci fossero tutti i dati a disposizione per poter procedere nello sviluppo in modo da rispettare le scadenze, riducendo le richieste in caso qualche ente privato non avesse voluto collaborare. In questo modo si sarebbe evitato di trascinare il progetto per un tempo tendenzialmente infinito. La speranza è che questa esperienza insegni qualcosa ai partecipanti al progetto e che queste vicende rimangano solamente brevi episodi nella vita di una stagista. 

\end{document}
